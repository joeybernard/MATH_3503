\documentclass[12pt,letterpaper, onecolumn]{exam}
\usepackage{amsmath}
\usepackage{amssymb}
\usepackage[lmargin=71pt, tmargin=1.2in]{geometry}  %For centering solution box
\lhead{Name\\}
\rhead{Student ID\\}
% \chead{\hline} % Un-comment to draw line below header
%\thispagestyle{empty}   %For removing header/footer from page 1

\begin{document}

\begingroup  
    \centering
    \LARGE MATH 3503 - Winter 2025\\
    \LARGE Quiz 3-B\\[0.5em]
    \large February 25, 2025\\[0.5em]
\endgroup
\rule{\textwidth}{0.4pt}
\pointsdroppedatright   %Self-explanatory
\printanswers
\renewcommand{\solutiontitle}{\noindent\textbf{Ans:}\enspace}   %Replace "Ans:" with starting keyword in solution box

\begin{questions}
    
    \question[3 Marks] When you have a repeated value of r when solving the characteristic equation of a second order ODE, why do you have to multiply the first solution by a power of x in order to get the second solution? What are you trying to ensure?\droppoints

    \bigskip
    \bigskip
    \bigskip
    \bigskip
    \bigskip
    \bigskip
    \bigskip
    \bigskip
    \bigskip
    \bigskip
    \bigskip
    \bigskip
    \bigskip
    \bigskip
    \bigskip
    \bigskip
    \bigskip
    \bigskip
    
    \question[3 Marks] Why do you check if an equation is exact by checking if $\frac{\partial M}{\partial y}$ is equal to $\frac{\partial N}{\partial x}$? Think about where M and N come from.\droppoints
    
    \pagebreak %Not necessary
\thispagestyle{empty}   %For removing header/footer from page 1
    
    \question[4 Marks] Find the fundamental matrix for the system of equations $Y^{'} = A Y$, where A is given by $\begin{bmatrix} 1 \; \; 1 \\ -1 \; \; 0 \end{bmatrix}$. You will need to solve this system first.\droppoints
    
\end{questions}
\end{document}
